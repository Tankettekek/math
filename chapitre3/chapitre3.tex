\documentclass[9pt,twoside]{article}

%Typesetting
\usepackage[a4paper, total={160mm,265mm}]{geometry}
\usepackage{titlesec}
\usepackage{amssymb}
\usepackage{float}
\usepackage{enumitem}
%Math packages
\usepackage{hyperref}
\usepackage{mathtools}
\usepackage{pgfplots}
\usepackage{tkz-tab}
\usepackage[french]{babel}

%PGFPLOTS options
\usepgfplotslibrary{fillbetween}
\pgfplotsset{holdot/.style={color=black,fill=white,only marks,mark=*}}

%Compat
\pgfplotsset{compat=1.18}

%Graph compilation optimization
%\usepgfplotslibrary{external}
%\tikzexternalize

%Typesetting definition
\titleformat{\subsubsection}[runin]
  {\normalfont\normalsize\bfseries}{\thesubsubsection}{1em}{}


%Definition d'un macro pour les definitions
\newcommand{\definition}[1]{\subsubsection*{\underline{Définition\textit{#1}}}}

%Title setup
\title{Chapitre 3: Compléments sur la dérivation et convéxité}
\author{George Alexandru Uzunov}
\date{}

\begin{document}

\maketitle
\tableofcontents\newpage


\section{Complément sur la dérivation}
\subsection{Etudier la dérivabilité en un point}
\subsubsection*{Exemple}
Soit une fonction $f$ telle que $f(x) = \begin{cases}x^2-2x-2 \text{ si } x\leq 1 \\
\frac{x-4}{x} \text{ si } x>1 \end{cases}$
\begin{itemize}[label=\textbullet]
	\item \textit{Etude de la continuité en 1:} $\lim_{x \to 1^{-}}(x^2-2x-2)=-3$ et $\lim_{x\to 1^{+}}\frac{x-4}{x}=-3$ \\ Les limites sont égales, il y a continuité.
	\item Pour $x\leq 1$, $f'(1^{-})=\lim_{h\to 0}\frac{f(1+h)-f(1)}{h}=\lim_{h\to 0}\frac{h^2}{h}=0$
	\item Pour $x>1$, $f(1^{+})=\lim_{h\to 0}\frac{f(1+h)-f(1)}{h}=\lim_{h\to 0}\frac{4}{1+h}=4$
\end{itemize}
Les limites sont différentes. Donc $f$ n'est pas dérivable en 1.\\ Pour $1^{-}$, la tangente horizontale est $y=-3$. \\ Pour $1^{+}$, la tangente horizontale est $y=4x+7$.

\subsection{Dérivée d'une fonction composée}
\subsubsection*{Propriété}
Soit une fonction $f$ telle que $x \xmapsto{} u(x) \xmapsto{} v[u(x)]$. Soit une fonction $u$ définie et dérivable en $I$, avec ses valeurs en $J$. Soit une fonction $v$ définie et dérivable en en $K$ tel que $J \subset K$, avec ses valeurs en $\mathbb{R}$. $f(x) = v\circ u(x)$ dérivable sur $I$. Sa dérivée est $f'(x) = v'\circ u \times u'$. 
\subsection{Dérivées usuelles de fonctions composées}
\begin{itemize}[label=\textbullet]
	\item Soit $u$ une fonction dérivable et strictement positive, avec $D_f \neq D_{f'}$: 
		$$f(x)=\sqrt{u(x)} \text{ et } f'(x)=\frac{u'(x)}{2\sqrt{u(x)}}$$
	\item Soit $u$ une fonction dérivable sur I, alors la fonction $f(x) = (u(x))^n$
		\begin{enumerate}
			\item est dérivable sur $I$ $\forall n\in \mathbb{N}^{*}$
			\item est dérivable sur $I$ si $u(x)\neq 0$ lorsque $n \in \mathbb{Z}^{*}$
		\end{enumerate}
		La dérivée de $f$ est alors $f'(x) = n \times u'(x) \times [u(x)]^{n-1}$
	\item Soit $u$ dérivable sur $I$, alors la fonction $f(x)=exp(u(x))$, sa dérivée est $f'(x)=u'(x)\times exp(u(x))$.
\end{itemize}
\subsection{Rappel de l'application de la dérivation}
La dérivée $f'$ sert à:
\begin{itemize}[label=\textbullet]
	\item Étudier la variation de la fonction $f$
	\item Étudier des extrema
		\begin{itemize}
			\item Lorsque $f'(x)$ est nulle
			\item Lorsqu'il y a changement de signe de la dérivée
		\end{itemize}
\end{itemize}
\subsection{Dérivée seconde}
\definition{}
Soit $f$ une fonction dérivable sur $I$ telle que sa dérivée $f'$ soit aussi dérivable sur $I$. On apelle dérivée seconde la fonction $f''(x)=(f'(x))'$.

\section{Convéxité}
\subsection{Lecture Graphique}
\definition{}
$f$ est convexe sur $I$ si et seulement si pour tout points a et b distincts de $C_f$, la corde $[AB]$ est au dessus de $C_f$. Elle est concave si et seulement si la corde $[AB]$ est en dessous de $C_f$.
\definition{ Point d'inflexion}
Le point $A$ de coordonées $(a;f(a))$ est un point d'inflexion de $C_f$ si et seulement si la tangente traverse la courbe au point A.
\subsection{Convéxité et sens de variation de \(f'\)}
\label{conv:deriv}
\subsubsection*{Propriété}
\begin{itemize}[label=\textbullet]
	\item $f$ est convexe sur $I$ si et seulement si $f'$ est croissante sur $I$.
	\item $f$ est concave sur $I$ si et seulement si $f'$ est décroissante sur $I$. 
\end{itemize}
\subsection{Convéxité et signe de \(f''\)}
\label{conv:deriv_sec}
\subsubsection*{Propriété 1}
\begin{itemize}[label=\textbullet]
	\item $f$ est convexe sur $I$ si et seulement si $f''$ est positive sur $I$.
	\item $f$ est concave sur $I$ si et seulement si $f''$ est négative sur $I$.
\end{itemize}
\subsubsection*{Propriété 2}
Soit $f$ une fonction deux fois dérivable sur $I$, $\forall x \in I$ on a: 
\begin{itemize}[label=\textbullet]
	\item si $f''(x)\geq 0$, alors $C_f$ est au dessus de ses tangentes.
	\item si $f''(x)\leq 0$, alors $C_f$ est en dessous de ses tangentes.	
\end{itemize} 
\subsubsection*{Démonstration}
On cherche à prouver que si $f''>0$ alors $C_f$ est au dessus de ses tangentes. 
\\(Méthode de la différence)
$$d(x) = f(x) - f'(a)(x-a)-f(a)$$
$$d'(x) = f'(x) - f'(a)$$
$$d''(x) = f''(x)$$
Or par hypothèse $f''(x)\geq 0$.
\begin{figure}[H]
	\centering
	\begin{tikzpicture}
	\tkzTabInit{$x$ /1,
			Signe de $d''(x)$ /1,
			Variation de $d'$ /1}
			{$-\infty$,$a$,$+\infty$}
	\tkzTabLine{,+,t,+}
	\tkzTabVar{-/,
		   R/,
		   +/}
	\tkzTabIma{1}{3}{2}{$0$}
	\end{tikzpicture}
	\caption{Tableau de signes de $d''(x)$ et tableau de variations de $d'$.}
\end{figure}
D'après ce tableau de signes on a:
\begin{itemize}[label=\textbullet]
	\item $\forall x \in ]-\infty;a[$, $d'(x)<0$.
	\item $\forall x \in [a;+\infty[$, $d'(x)\geq 0$.
\end{itemize}
On a donc:
\begin{figure}[H]
	\centering
	\begin{tikzpicture}
	\tkzTabInit{$x$ /1,
			Signe de $d'(x)$ /1,
			Variation de $d$ /1}
			{$-\infty$,$a$,$+\infty$}
	\tkzTabLine{,-,z,+}
	\tkzTabVar{+/,
		   -/$0$,
		   +/}
	\end{tikzpicture}
	\caption{Tableau de signes de $d'(x)$ et tableau de variations de $d$.}
\end{figure}
L'extremum est 0 pour $d$. Donc $d(x)\geq 0$.
$C_f$ est donc toujours au dessus de ses tangentes.
\subsection{Point d'inflexion et dérivée seconde}
\subsubsection*{Propriété}
Soit $f$ une fonction deux fois dérivable sur $I$. Le point $A(a;f(a))$ est un point d'inflexion si et seulement si $f''$ s'annule en $a$ tout en changeant de signe.
\subsection{Synthèse}
\begin{enumerate}
	\item Les trois propositions suivantes sont équivalentes. $f$ est convexe sur $I$ $\iff$ $f'$ est croissante sur $I$ $\iff$ $f''\geq 0$.
	\item Les trois propositions suivantes sont équivalentes. $f$ est concave sur $I$ $\iff$ $f'$ est décroissante sur $I$ $\iff$ $f''\leq 0$.
\end{enumerate} \newpage
\section{Annexe}
\begin{figure}[h]
	\centering
	\begin{tikzpicture}
		\begin{axis}[axis y line=middle,axis x line=middle,ticks=none]
                 	\addplot[ 
	                        domain=-3:3,
                       		samples=1000,
        	                color=black,
        	        ]
			{x^3};
        	        \addlegendentry{$f(x)=x^3$}
			\addplot[ 
	                        domain=-3:3,
                       		samples=1000,
        	                color=black,
				densely dotted,
				thick,
				]
			{3*x^2};
        	        \addlegendentry{$f''(x)=3x^2$}
			\addplot[ 
	                        domain=-3:3,
                       		samples=1000,
        	                color=black,
				densely dashed,
				thick,
				]
			{6*x};
        	        \addlegendentry{$f''(x)=6x$}
        	\end{axis}
        \end{tikzpicture}
	\caption{Courbe d'une fonction, sa dérivée et sa derivée seconde}
\end{figure}
Dans cette représentation on peut clairement observer le signe de $f''(x)$, la variation de $f'$ et la convéxité de $f$. Cette figure illustre les propriétés exposées dans les subsections \ref{conv:deriv} et \ref{conv:deriv_sec}. 
\end{document}
