\documentclass{article}

%Typesetting packages
\usepackage[a4paper, total={160mm,265mm}]{geometry}
\usepackage{float}
\usepackage{titlesec}
\usepackage{enumitem}
\usepackage[framemethod=pstricks]{mdframed}

%Fonts
\usepackage{fontspec}
\setmainfont{Roboto}



%Typesetting of elements
%%Subsubsections
\titleformat{\subsubsection}[runin]
  {\normalfont\normalsize\bfseries}{\thesubsubsection}{1em}{}
\titlespacing{\subsubsection}{0em}{0.2em}{1em}


%Math packages
\usepackage{mathtools}
\usepackage{pgfplots}
\usepackage{tkz-tab}
\usepackage{amssymb}

%PGFPLOTS options
\usepgfplotslibrary{fillbetween}
%Compat
\pgfplotsset{compat=1.18}
%PGFPLOTS macro
\pgfplotsset{holdot/.style={color=black,fill=white,only marks,mark=*}}
\usepgfplotslibrary{external}
\tikzexternalize[prefix=pgfplotsfigures/]

%macro pour les definitions
\newcommand{\definition}[1]{
	\subsubsection*{\underline{Définition\textit{#1}}}
}

%Box (defbox)
\newmdenv[linecolor=black, linewidth=0.7pt, backgroundcolor=lightgray,innertopmargin=5pt, innerbottommargin=5pt, roundcorner=5pt]{defbox}


%Language package
\usepackage[french]{babel}

\title{Chapitre 4: Successions d'épreuves indépendantes}
\author{George Alexandru Uzunov}
\date{}

%%Paramètres graphiques pour représenter l'arbre de probabilités
\tikzstyle{level 1}=[level distance=3.5cm, sibling distance=3.5cm]
\tikzstyle{level 2}=[level distance=3.5cm, sibling distance=2cm]
\tikzstyle{bag} = [text width=4em, text centered]

\begin{document}
\maketitle
\tableofcontents\newpage

\section{Représenter une succession d'epreuves}

\subsection{Rappel sur l'arbre pondeéré}

\begin{figure}[H]
	\centering
	\begin{tikzpicture}[grow=right, sloped]
		\node[bag]{$\Omega$}
			child{
				node[bag]{$S$}
					child{
						node[bag]{$A$}
						edge from parent
						node[below]{$P_2$}
					}
					child{
						node[bag]{$\overline{A}$}
						edge from parent
						node[above]{$1-P_2$}
					}
					edge from parent
					node[below]{$P_1$}
				}
			child{
				node[bag]{$\overline{S}$}
					child{
						node[bag]{$A$}
						edge from parent
						node[below]{$P_2$}
					}
					child{
						node[bag]{$\overline{A}$}
						edge from parent
						node[above]{$1-P_2$}
					}
					edge from parent
					node[above]{$1-P_1$}
				};
	\end{tikzpicture}
	\caption{Représentation d'un arbre pondéré}
\end{figure}

\subsection{Successions d'épreuves indépendantes}

\begin{defbox}
	\sloppy
	\definition{}
	Dans une succession d'épreuves, lorsque l'issue d'une épreuve ne dépend pas des épreuves précédentes, on dit qu'elle est indépendante.
\end{defbox}

\subsubsection*{Propriétés}
Lorsqu'on répète $n$ fois de façon indépendante une éxpérience aléatoire dont les issues sont $A_1, A_2, \dots, A_n$ pour lesquelles les probabilités sont $P(A_1), P(A_2), \dots, P(A_n)$, alors la probabilité d'obtenir la suite d'issues $A_1$ jusqu'à $A_n$ est le produit de leur probabilités. 
\subsubsection*{Exemple}
Soit un dé à quatre faces équilibré. Si ce dé est numéroté de 1 à 4:\\\\
\begin{centering}
	\begin{tabular}{|c|c|c|c|c|}
		\hline
		$x$ & 1 & 2 & 3 & 4 \\ \hline
		$P(x)$ & $\frac{1}{4}$ & $\frac{1}{4}$ & $\frac{1}{4}$ & $\frac{1}{4}$ \\ \hline
	\end{tabular}
\end{centering}\\\\
De même, soient un jeton $A$ et deux jetons $B$ placés dans un sac. Si, successivement nous lançons le dé puis nous tirons un jeton, les issues sont les suivantes.
% Arbre pondéré
%\begin{figure}
%	\centering
%		\begin{tikzpicture}[grow=right, sloped]
%		
%		\end{tikzpicture}	
%	\caption{Arbre pondéré des issues possibles}
%\end{figure}
Ceci donne la loi de probabilité suivante:\\\\
\begin{centering}
	\begin{tabular}{|c|c|c|c|c|c|c|c|c|}
		\hline
		Issues & $(1;A)$ & $(1;B)$ & $(2;A)$ & $(2;B)$ & $(3;A)$ & $(3;B)$ & $(4;A)$ & $(4;B)$ \\ \hline
		Probabilité de chaque issue & $\frac{1}{12}$ &$\frac{1}{6}$&$\frac{1}{12}$&$\frac{1}{6}$ &  $\frac{1}{12}$&$\frac{1}{6}$ & $\frac{1}{12}$ & $\frac{1}{6}$\\ \hline
	\end{tabular}
\end{centering}


\section{Loi de Bernoulli}


\section{Loi Binomiale}

\subsection{Schema de Bernoulli}

\subsection{Etude d'un exemple}

\subsection{Coefficients binomiaux}


\section{Application de la loi binomiale}

\end{document}
