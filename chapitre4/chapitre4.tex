\documentclass{article}
\usepackage{typeset}
\usepackage[french]{babel}

\title{Chapitre 4: Successions d'épreuves indépendantes}
\author{George Alexandru Uzunov}
\date{}

%%Paramètres graphiques pour représenter l'arbre de probabilités
\tikzstyle{bag} = [text width=4em, text centered]

\begin{document}
\maketitle
\tableofcontents\newpage

\section{Représenter une succession d'epreuves}

\subsection{Rappel sur l'arbre pondeéré}

\begin{figure}[H]
	\centering
	\begin{tikzpicture}[grow=right, sloped]
		\node[bag]{$\Omega$}
			child{
				node[bag]{$S$}
					child{
						node[bag]{$A$}
						edge from parent
						node[below]{$P_2$}
					}
					child{
						node[bag]{$\overline{A}$}
						edge from parent
						node[above]{$1-P_2$}
					}
					edge from parent
					node[below]{$P_1$}
				}
			child{
				node[bag]{$\overline{S}$}
					child{
						node[bag]{$A$}
						edge from parent
						node[below]{$P_2$}
					}
					child{
						node[bag]{$\overline{A}$}
						edge from parent
						node[above]{$1-P_2$}
					}
					edge from parent
					node[above]{$1-P_1$}
				};
	\end{tikzpicture}
	\caption{Représentation d'un arbre pondéré}
\end{figure}

\subsection{Successions d'épreuves indépendantes}

\begin{defbox}
	\sloppy
	\definition{}
	Dans une succession d'épreuves, lorsque l'issue d'une épreuve ne dépend pas des épreuves précédentes, on dit qu'elle est indépendante.
\end{defbox}

\subsubsection*{Propriétés}
Lorsqu'on répète $n$ fois de façon indépendante une éxpérience aléatoire dont les issues sont $A_1, A_2, \dots, A_n$ pour lesquelles les probabilités sont $P(A_1), P(A_2), \dots, P(A_n)$, alors la probabilité d'obtenir la suite d'issues $A_1$ jusqu'à $A_n$ est le produit de leur probabilités. 
\subsubsection*{Exemple}
Soit un dé à quatre faces équilibré. Si ce dé est numéroté de 1 à 4:
\begin{figure}[H]
	\centering
	\begin{tabular}{|c|c|c|c|c|}
		\hline
		$x$ & 1 & 2 & 3 & 4 \\ \hline
		$P(x)$ & $\sfrac{1}{4}$ & $\sfrac{1}{4}$ & $\sfrac{1}{4}$ & $\sfrac{1}{4}$ \\ \hline
	\end{tabular}
\end{figure}
De même, soient un jeton $A$ et deux jetons $B$ placés dans un sac. Si, successivement nous lançons le dé puis nous tirons un jeton, les issues sont les suivantes.
\begin{figure}[H]
	\centering
	\begin{forest}
		ptree
		[$\Omega$
			[$1$:\sfrac{1}{4}
				[$A$:\sfrac{1}{3}]
				[$B$:\sfrac{2}{3}]
			]
			[$2$:\sfrac{1}{4}
				[$A$:\sfrac{1}{3}]
				[$B$:\sfrac{2}{3}]
			]
			[$3$:\sfrac{1}{4}
				[$A$:\sfrac{1}{3}]
				[$B$:\sfrac{2}{3}]
			]
			[$4$:\sfrac{1}{4}
				[$A$:\sfrac{1}{3}]
				[$B$:\sfrac{2}{3}]
			]
		]
	\end{forest}
	\caption{}
\end{figure}

Ceci donne la loi de probabilité suivante:
\begin{figure}[H]
	\centering
	\begin{tabular}{|c|c|c|c|c|c|c|c|c|}
		\hline
		Issues & $(1;A)$ & $(1;B)$ & $(2;A)$ & $(2;B)$ & $(3;A)$ & $(3;B)$ & $(4;A)$ & $(4;B)$ \\ \hline
		Probabilité de chaque issue & $\sfrac{1}{12}$ &$\sfrac{1}{6}$&$\sfrac{1}{12}$&$\sfrac{1}{6}$ &  $\sfrac{1}{12}$&$\sfrac{1}{6}$ & $\sfrac{1}{12}$ & $\sfrac{1}{6}$\\ \hline
	\end{tabular}
\end{figure}


\section{Loi de Bernoulli}
\begin{defbox}
	\sloppy
	\definition{}
	Une épreuve de Bernoulli est une éxperience aléatoire qui admet exactement deux issues possibles (succès '$S$', échec '$E$) 
\end{defbox}
\begin{defbox}
	\sloppy
	\definition{}
	Soit $X$ la variable aléatoire prenant la valeur 1 si S est réalisé avec une probabilité $p$ la loi de probabilité est:
	\begin{figure}[H]
		\centering
		\begin{tabular}{|c|c|c|}
			\hline
			$k$ & 0 & 1 \\ \hline
			$P(X=k)$ & $1-p$ & $p$ \\ \hline	
		\end{tabular}
		\caption{Loi de Bernoulli de paramètre $p$}
	\end{figure}
\end{defbox}
\subsubsection*{Propriétés}
$$E(X) = 0(1-p)+ 1\times p = p$$
$$V(X) = p-p^2 = p(1-p)$$
$$\sigma(X) = \sqrt{p(1-p)}$$
\section{Loi Binomiale}

\subsection{Schema de Bernoulli}

\begin{defbox}
	\sloppy
	\definition{}
	On apelle Schema de Bernoulli d'ordre $n$ la répétition de $n$ épreuves de Bernoulli \underline{\underline{identiques et indépendantes}}.
\end{defbox}

\subsection{Etude d'un exemple}
Soit un schéma de Bernoulli d'ordre 3.

\begin{figure}[H]
	\centering
	\begin{forest}
		ptree
		[$\Omega$
			[$S$:p
				[$S$:p
					[$S$:p][$\overline{S}$:1-p]
				]
				[$\overline{S}$:1-p
					[$S$:p][$\overline{S}$:1-p]
				]
			]
			[$\overline{S}$:1-p
				[$S$:p
					[$S$:p][$\overline{S}$:1-p]
				]
				[$\overline{S}$:1-p
					[$S$:p][$\overline{S}$:1-p]
				]
			]
		]
	\end{forest}
\end{figure}
Soit $X$ la variable aléatoire modelisant le nombre ($k$) de succès. On a: $X(\Omega) = {0;1;2;3}$
\subsection{Coefficients binomiaux}


\section{Application de la loi binomiale}

\end{document}
