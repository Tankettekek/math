\documentclass{article}
\usepackage{typeset}
\usepackage[french]{babel}

\title{Continuité}
\author{George Alexandru Uzunov}
\date{}

\begin{document}
\maketitle
\tableofcontents\newpage
\section{Notion de continuité}
\begin{defbox}
	\sloppy
	\definition{}
	Soit une fonction $f$ définie sur un intervalle $I$ et $a \in I$.
	$f$ est continue en $a$ si $\lim_{x \to a} f(x) = f(a)$.
	$$
	\begin{cases}
		\lim_{x \to a^{-}} f(x) = f(a) \\
		\lim_{x \to a^{+}} f(x)  = f(a)	
	\end{cases}
	$$
	\begin{figure}[H]
		\centering
		\begin{tikzpicture}
			\begin{axis}[
				xlabel={$x$},
				ylabel={$y$},
				axis lines=center,
				xmin = -1, xmax = 3,
				ymin = -1, ymax = 22,
				xtick = {1.5},
				ytick = {4.48},
				xticklabels={$a$},
				yticklabels={$f(a)$},
				]
				\addplot[domain=-1:3,
					samples=100,
					color=black]{e^x};
				\node[fill,circle,inner sep=1pt] at (1.5,4.48){};
				\draw[dashed,color=black] (0,4.48) -| (1.5,0);
			\end{axis}
		\end{tikzpicture}
		\caption{$f$ est continue sur $D_f$}
	\end{figure}
\end{defbox}
\subsubsection*{Exemple}
\begin{enumerate}
	\item La fonction valeur absolue ($f:x \mapsto |x|$): 
		\begin{itemize}
			\item Continue
			\item Non derivable en 0
		\end{itemize}
	\item La fonction racine carree ($f:x \mapsto \sqrt{x}$)
		\begin{itemize}
			\item $D_f = \mathbb{R}^{*}$ \,\,\,\,\, $D_{f^{'}} = \mathbb{R}_{+}^{*}$
			\item Continue en 0
			\item Non derivable en 0
		\end{itemize}
	\item La fonction exponentielle ($f:x \mapsto e^x$)
		\begin{itemize}
			\item Continue
			\item Derivable
		\end{itemize}
	\item La fonction partie entiere ($f:x \mapsto \lfloor x \rfloor$)
		\begin{itemize}
			\item Non continue
			\item Non derivable
		\end{itemize}
\end{enumerate}
\section{Continuite des fonctions usuelles}
\begin{figure}[H]
	\centering
	\scalebox{1.2}{%
		\begin{tabular}{|c|c|}
			\hline
			Fonction & Intervalle de continuite \\ \hline\hline
			$x^n$ (avec $n \in \mathbb{N}$) & $\mathbb{R}$ \\ \hline
			$\frac{1}{x^n}$ & $\mathbb{R}^{*}$ \\ \hline
			$\sqrt{x}$ & $\mathbb{R}^{*}$ \\ \hline
			$|x|$ & $\mathbb{R}$ \\ \hline
			$e^x$ & $\mathbb{R}$ \\ \hline
			$\sin(x)$ & $\mathbb{R}$ \\ \hline
			$\cos(x)$ & $\mathbb{R}$ \\ \hline
		\end{tabular}
	}
	\caption{Tableau de continuite des fonctions de reference}
\end{figure}	


	Toute fonction construite par somme, produit, quotient ou composition à partir des fonctions de reference herite de la continuite des fonctions de reference utilisees.\\

	\subsubsection*{Exemple}
	$$ f:x \mapsto x^2+1 \mapsto \cos (x^2+1) \mapsto e^{\cos (x^2+1)} $$
	Par theoreme de continuite des fonctions des fonctions de reference, par somme et par composition, on peut conclure que $f$ est continue sur $\mathbb{R}$. \\ 
	\subsubsection*{Exercice type}
	$$f(x) = 
	\begin{cases}
		-x+2 \text{ pour } x < 3\\
		x - 4 \text{ pour } 3 \leq x < 5 \\
		-2x+13 \text{ pour } x\geq 5
	\end{cases}
	$$
	La fonction est elle continue?
	
	$$ 	\begin{rcases*}
			\lim_{x \to 3^{-}} -x+2 = -1	\\
			\lim_{x \to 3^{+}} x-4 = -1 \\
		\end{rcases*} \text{ continue en 3}
	$$
	$$
		\begin{rcases*}
			\lim_{x \to 5^{-}} x-4  = 1 \\
			\lim_{x \to 5^{+}} -2x+13 = 3
		\end{rcases*} \text{ pas continue en 5}
	$$
\section{Continuite et derivabilite}
\begin{defbox}
	\sloppy
	\subsubsection*{\underline{Theoreme \textit{(admis)}}}
	\begin{itemize}
		\item Si $f$ est derivable en $a$ alors la fonction $f$ est continue en $a$.
		\item Si $f$ est derivable sur $I$, alors elle est continue sur cet intervalle.
	\end{itemize}
\end{defbox}
\section{Continuite et equations}
\begin{defbox}
	\sloppy
	\subsubsection*{\underline{Theoreme \textit{(admis)}}}
	Soit une fonction continue sur un intervalle $I = [a;b]$, pour tout $k$ tel que\\ $f(a) \leq k \leq f(b)$, l'equation $f(x) = k$ admet au moins une solution dans $I$.
	\\
	\subsubsection*{Remarque}
	Le theoreme permet de prouver qu'il existe au moins une solution.
	\\
	\subsubsection*{\underline{Corrolaire du theoreme}}
	Soit une fonction $f$ continue et strictement monotone sur $I$. Soit $f(a)\leq k \leq f(b)$, l'equation $f(x) = k$ admet une unique solution en $I$.
\end{defbox}
\subsubsection*{Exemple}
$$f: \begin{cases}
	\mathbb{R} \to \mathbb{R}\\
	x \mapsto x^3-3x^2 -1
\end{cases}$$
Quel est le nombre de solutions a l'equation $f(x)=4$ sur $\mathbb{R}$?
$$f'(x)=3x^2-6x$$
On cherche les racines de la derivee.
$$3x^2-6x=0$$
$$\Delta = 36$$
D'où: $x_1 = 2$ et $x_2 = 0$.

\begin{figure}[H]
	\centering
	\begin{tikzpicture}
	\tkzTabInit{$x$ /1,
			Signe de $f'(x)$ /1,
			Variation de $f$ /1}
			{$-\infty$,$0$,$2$,$+\infty$}
	\tkzTabLine{,+,z,-,z,+}
	\tkzTabVar{-/,
print(recherche("AGTC", "GTACAAATCTTGCC")) 
		   +/$-1$,
		   -/$-5$,
	   	   +/}
	\end{tikzpicture}
	\caption{Tableau de signes de $f'(x)$ et tableau de variations de $f$.}
\end{figure}
Sur $]-\infty;0]$ 0 solutions\\
\indent Sur $[0;2]$ 0 solutions\\
\indent Sur $[2;+\infty]$ 1 solution.

\section{Continuite et suites}
\begin{defbox}
	\sloppy
	\subsubsection*{\underline{Theoreme (du point fixe)}}
	Soit une suite $(U_n)$ definie par la relation $U_{n+1} = f(U_n)$ convergente vers $l$. Si la fonction associee $f$ est continue sur $\mathbb{R}$, alors la limite de la suite en $N$ est solution de l'equation $f(x)=x$.\\
	\indent Sachant que :
	\begin{itemize}
		\item $\forall n \in \mathbb{N}$,\,\, $U_n \in I$ fermé
		\item $f$ continue en $l$,  $l\in I$ ou $f$ a valeur dans $I$ ou $I$ contient tous les termes de $(U_n)$ ou $f(I) \subset I$ ou sachant que $x\in I$, \,\, $f(x) \in I$.
	\end{itemize}
\end{defbox}
\subsubsection*{Exemple}
Soit 
$$ (U_n) = \begin{cases}
	U_0 = 1 \\
	U_{n+1} = \frac{3}{U_n+1}
\end{cases}
$$
Determinez la limite de $(U_n)$.

\paragraph{1.} Posons $U_{n+1} = f(U_n)$ où $f(x) = \frac{3}{x+1}$ sur $I=[0;3]$. On admet que$(U_n)$ converge et que ses valeurs sont contenues dans l'intervalle $[0;3]$
\paragraph{2. Continuité} $f$ est continue sur $I$ car elle est la fonction inverse d'une fonction continue ne s'annulant pas sur $I$.
\paragraph{3.} Verifions que toutes les images de $x$ par $f$ appartiennent à $I$.

\begin{equation*}
\begin{split}
	0 \leq x \leq 3 \\
	\iff 1 \leq x+1 \leq 4 \\
	\iff 1 \geq \frac{1}{x+1} \geq \frac{1}{4} \\
	\iff 3 \geq \frac{3}{x+1} \geq \frac{3}{4} \\
\end{split}
\end{equation*}
\paragraph{4.} $(U_n)$ converge vers $l$, $f$ est continue et à valeur dans $I$ donc d'après le theoreme du point fixe on a $f(x)=x$.

\begin{equation*}
\begin{split}
	f(x)=x \iff  \frac{3}{x+1} = x \iff x^2+x-3 = 0	\,\,\,\, \Delta = 13\\
	x_1 = \frac{1 - \sqrt{13}}{2} \,\,\,\, \text{et} \,\,\,\, x_2 = \frac{-1- \sqrt{13}}{2} \,\,\, \text{or} \,\,\, x_2 \notin I \,\,\, \text{donc} \,\,\, \lim_{n \to + \infty} (U_n) = \frac{-1+\sqrt{13}}{2} 
\end{split}
\end{equation*}


\end{document}
